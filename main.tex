\input{./preamble.tex}


\title{GSI Green Cube}
    \subtitle{\textit{High-Performance Computing in Nuclear Physics Research}}
\author{Tobiasz Fic}
\date{19 May 2024}

\begin{document}

\begin{frame}
    \maketitle
\end{frame}

\begin{frame}{GSI and FAIR}
    \begin{columns}
        \begin{column}{0.5\textwidth}
            \begin{figure}
                \centering
                \includegraphics[width=\textwidth]{images/fair_sis100_diagram.jpg}
            \end{figure}
        \end{column}
        \begin{column}{0.5\textwidth}
            \begin{itemize}
                \item GSI (Gesellschaft für Schwerionenforschung) Helmholtzzentrum für Schwerionenforschung is a nuclear research facility in Darmstadt, Germany.
                \item The FAIR particle accelerator  facility is under construction (due 2027).
            \end{itemize}
        \end{column}
    \end{columns}
\end{frame}


\begin{frame}{Need for High-Performance Computing}
    \begin{columns}
        \begin{column}{0.5\textwidth}
            \begin{itemize}
                \item The rising complexity of nuclear physics experiments requires more computational power.
                \item Example: the CBM experiment will have an interaction rate of \SI{10}{\mega\hertz}.
            \end{itemize}
        \end{column}
        \begin{column}{0.5\textwidth}
            \begin{figure}
                \centering
                \includegraphics[width=\textwidth]{images/galatyuk_map_of_experiments.png}
                \includegraphics[width=0.2\textwidth]{images/cbm_logo.png}
            \end{figure}
        \end{column}
    \end{columns}
\end{frame}

\begin{frame}{Green Cube Data Center}
    \begin{columns}
        \begin{column}{0.5\textwidth}
            \begin{figure}
                \centering
                \includegraphics[width=\textwidth]{images/gsi_green_cube.jpg}
            \end{figure}
        \end{column}
        \begin{column}{0.5\textwidth}
            \begin{itemize}
                \item The Green Cube was built in 2015 as a solution to the growing computational needs of GSI (and in the future, FAIR).
                \item The investment totaled \num{12} million euros in construction costs and \num{21} million euros overall.
            \end{itemize}
        \end{column}
    \end{columns}
\end{frame}

% overall specs
\begin{frame}{Green Cube: Overall Specs}
    \begin{columns}
        \begin{column}{0.5\textwidth}
            \begin{itemize}
                \item size: 27 \si{\meter} x 30 \si{\meter} x height 22 \si{\meter} (6 levels)
                \item 12 \si{\mega\watt} cooling capability
                \item 600 nodes, on average 90 cores per node
                \item 400 GPUs
            \end{itemize}
        \end{column}
        \begin{column}{0.5\textwidth}
            \begin{itemize}
                \item Maximum number of racks: 768 2.2 \si{\meter} racks
                \item 60 petabytes of storage, will be expanded to 250 petabytes for FAIR
            \end{itemize}
        \end{column}
    \end{columns}
\end{frame}


% storage capabilities

% lustre
\begin{frame}{Software: \textit{Lustre} File System}
    \begin{columns}
        \begin{column}{0.4\textwidth}
            \begin{figure}
                \centering
                \includegraphics[width=\textwidth]{images/lustre_logo.png}
            \end{figure}
        \end{column}
        \begin{column}{0.6\textwidth}
            \begin{itemize}
                \item High-performance distributed file system.
                \item Open-source, POSIX-compliant.
                \item Used in many supercomputers and data centers.
                \item Scalable to thousands of devices.
            \end{itemize}
        \end{column}
    \end{columns}
\end{frame}

\begin{frame}[fragile]{Lustre: Mounting the File System Remotely}
    \begin{columns}
        \begin{column}{0.4\textwidth}
            \begin{figure}
                \centering
                \includegraphics[width=\textwidth]{images/lustre_petabytes.png}
            \end{figure}
        \end{column}
        \begin{column}{0.6\textwidth}
            In a file, such as \texttt{mount.sh}:
            \begin{lstlisting}[language=bash]
mnt='/tmp/lustre/'
mkdir -p $mnt
sshfs -o ProxyJump=lxpool.gsi.de tfic@lustre.hpc.gsi.de:/lustre $mnt
            \end{lstlisting}
            then, to mount the file system:
            \begin{lstlisting}[language=bash]
bash mount.sh
            \end{lstlisting}
            to unmount:
            \begin{lstlisting}[language=bash]
fusermount -u $mnt
            \end{lstlisting}
        \end{column}
    \end{columns}
\end{frame}

% working nodes

% slurm workload manager
\begin{frame}{Software: \textit{SLURM} Workload Manager}
    \begin{columns}
        \begin{column}{0.4\textwidth}
            \begin{figure}
                \centering
                \includegraphics[width=0.5\textwidth]{images/slurm_logo.png}
            \end{figure}
        \end{column}
        \begin{column}{0.6\textwidth}
            \begin{itemize}
                \item Workload manager for Linux clusters.
                \item Open-source.
                \item Used in many supercomputers and data centers.
                \item Scalable to thousands of devices.
                \item Easy to use: \texttt{sbatch}, \texttt{srun}, \texttt{squeue}, \texttt{sinfo}.
                      Work gets delegated to nodes automatically through bash scripts or commands.
            \end{itemize}
        \end{column}
    \end{columns}
\end{frame}

\begin{frame}[fragile]{SLURM: Example Job Script}
    \begin{columns}
        \begin{column}{0.4\textwidth}
            \begin{figure}
                \centering
                \includegraphics[width=0.5\textwidth]{images/slurm_logo.png}
            \end{figure}
        \end{column}
        \begin{column}{0.6\textwidth}
            In a file, such as \texttt{job.sh}:
            \begin{lstlisting}[language=bash]
#!/bin/bash
#SBATCH --output %j_%N.out
hostname ; sleep ${1:-30}
            \end{lstlisting}
            then, to launch the script as a job:
            \begin{lstlisting}[language=bash]
sbatch job.sh
            \end{lstlisting}
            to check job status:
            \begin{lstlisting}[language=bash]
squeue -u $USER 
            \end{lstlisting}
            to cancel a job:
            \begin{lstlisting}[language=bash]
scancel <job_id>
            \end{lstlisting}
        \end{column}
    \end{columns}
\end{frame}

\begin{frame}{}
    \centering
    \Large{Thank you for your attention}
\end{frame}

\end{document}
